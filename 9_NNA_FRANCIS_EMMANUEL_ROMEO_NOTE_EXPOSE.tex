\documentclass[12pt,a4paper]{article}
\usepackage[utf8]{inputenc}
\usepackage[T1]{fontenc}
\usepackage[french]{babel}
\usepackage{geometry}
\usepackage{hyperref}
\usepackage{setspace}
\usepackage{titlesec}
\usepackage{enumitem}
\usepackage{graphicx}
\usepackage{booktabs}
\geometry{margin=2.5cm}
\setstretch{1.15}

\title{\textbf{RÉSUMÉ TECHNIQUE DES EXPOSÉS EN INVESTIGATION NUMÉRIQUE}}
\author{NNA FRANCIS EMMANUEL ROMEO \\ Matricule : 22P056 \\ Filière : CIN-4}
\date{Octobre 2025}

\begin{document}
\maketitle
\thispagestyle{empty}
\begin{center}
\textbf{--- Synthèse technique et thématique ---}
\end{center}

\section*{1. Reconnaissance Faciale et Intelligence Artificielle}
La reconnaissance faciale est une technologie biométrique issue de l’intelligence artificielle permettant d’identifier ou de vérifier une identité à partir des caractéristiques morphologiques du visage. Elle repose sur quatre modules principaux :
\begin{itemize}[noitemsep]
  \item \textbf{Acquisition} : capture d’images ou de flux vidéo à partir de capteurs.
  \item \textbf{Extraction de caractéristiques} : traitement d’images via CNN (Convolutional Neural Networks).
  \item \textbf{Correspondance} : comparaison des vecteurs de traits dans un espace euclidien ou cosinus.
  \item \textbf{Décision} : classification supervisée selon un seuil de similarité ($\theta$).
\end{itemize}
Les approches actuelles combinent méthodes globales et locales via des architectures hybrides (FaceNet, ArcFace). Cette technologie joue un rôle clé dans les enquêtes forensiques et la surveillance intelligente.

\bigskip
\section*{2. Falsification de Messages Numériques et Chaîne de Preuve}
Les échanges numériques (WhatsApp, Messenger, Telegram) constituent aujourd’hui des sources majeures de preuves judiciaires. Cependant, leur authenticité peut être compromise par :
\begin{itemize}[noitemsep]
  \item \textbf{Manipulation de captures d’écran} à l’aide d’outils comme Chatsmock, Photoshop ou HTMLForge.
  \item \textbf{Injection de faux messages} dans des bases SQLite d’applications mobiles.
\end{itemize}
\textbf{Mesures préventives :}
\begin{itemize}[noitemsep]
  \item Vérification des métadonnées et hachage (SHA-256).
  \item Corrélation entre horodatages et journaux système.
  \item Analyse de la signature numérique et de la source d’origine.
  \item Formation des magistrats et enquêteurs aux preuves numériques.
\end{itemize}
Ces étapes garantissent la traçabilité et l’intégrité de la preuve dans le respect du \textit{Digital Evidence Standard} (Eoghan Casey).

\bigskip
\section*{3. Deepfake et Détection Forensique}
Les \textbf{deepfakes} (faux profonds) exploitent des GANs (Generative Adversarial Networks) pour générer des vidéos ou audios ultra-réalistes. 
\begin{itemize}[noitemsep]
  \item \textbf{Applications légitimes} : doublage automatique, accessibilité vocale, préservation patrimoniale.
  \item \textbf{Applications malveillantes} : usurpation d’identité, fraude vocale, désinformation.
\end{itemize}
\textbf{Techniques de détection :}
\begin{itemize}[noitemsep]
  \item Analyse spectrale et détection de jitter dans les signaux audio.
  \item Modèles de classification CNN-RNN entraînés sur DFDC et Celeb-DF.
  \item Intégration de filigranes numériques (watermarking quantique).
\end{itemize}
Le deepfake vocal illustre la convergence entre IA, cybersécurité et investigation numérique.

\bigskip
\section*{4. Archéologie des Régimes de Vérité Numérique}
L’\textbf{archéologie numérique} foucaldienne analyse les discontinuités entre anciens et nouveaux régimes de vérité. 
\[
\vec{R_t} = (\alpha_T, \alpha_J, \alpha_S, \alpha_P)
\]
\begin{itemize}[noitemsep]
  \item $\alpha_T$ : dominance technologique (plateformes, IA)
  \item $\alpha_J$ : dominance juridique (lois, RGPD)
  \item $\alpha_S$ : dominance scientifique (expertise, vérification)
  \item $\alpha_P$ : dominance politique/médiatique (influence, opinion)
\end{itemize}
De 1990 à 2020, la transition s’opère d’un régime \emph{expert-centrique} vers un régime \emph{plateforme-centré}, marqué par l’autorité algorithmique. Les discontinuités épistémologiques traduisent une \textbf{mutation socio-technique progressive mais disruptive}.

\bigskip
\section*{5. Cryptographie Post-Quantique et Protocole ZK-NR}
Le protocole \textbf{ZK-NR (Zero-Knowledge Non-Repudiation)} vise à garantir la non-répudiation tout en préservant la confidentialité. 
\begin{itemize}[noitemsep]
  \item \textbf{Architecture} : combinaison de Merkle Trees, signatures BLS à seuil et preuves STARKs.
  \item \textbf{Complexité} : génération de preuve $O(n\log n)$, vérification $O(1)$.
  \item \textbf{Sécurité} : résistance post-quantique (Dilithium, FALCON, SPHINCS+).
\end{itemize}
Ce modèle permet la vérification sans divulgation, ouvrant la voie à une nouvelle \textbf{chaîne de custody cryptographique} dans les enquêtes numériques modernes.

\bigskip
\section*{6. Cybercriminalité Africaine et Investigation Numérique}
L’Afrique fait face à une explosion des menaces cybernétiques :
\begin{itemize}[noitemsep]
  \item \textbf{Ransomwares et fraudes bancaires} (Mobile Money, fintech).
  \item \textbf{Espionnage numérique étatique} et cyberactivisme.
\end{itemize}
\textbf{Facteurs de vulnérabilité :}
\begin{itemize}[noitemsep]
  \item Faible maturité législative et institutionnelle.
  \item Dépendance à l’hébergement étranger.
  \item Pénurie d’experts certifiés en forensic (moins d’un pour 100 000 hab.).
\end{itemize}
\textbf{Méthodologie forensique standardisée :}
\begin{enumerate}[noitemsep]
  \item Identification de l’incident.
  \item Acquisition de preuves (images disques, journaux systèmes).
  \item Préservation de l’intégrité (hachage, scellé numérique).
  \item Analyse technique (Autopsy, FTK, Wireshark).
  \item Rédaction du rapport et archivage probatoire.
\end{enumerate}

\bigskip
\section*{7. Intelligence Artificielle Générative et Enjeux Probatoires}
L’IA générative (ChatGPT, DALL·E, Synthesia) redéfinit la production de contenu et pose de nouveaux défis juridiques :
\begin{itemize}[noitemsep]
  \item \textbf{Traçabilité des modèles} : absence de logs explicites de génération.
  \item \textbf{Authenticité des preuves synthétiques} : difficulté à prouver la source.
  \item \textbf{Attribution} : nécessité de protocoles de marquage (watermark AI).
\end{itemize}
Les systèmes d’IA deviennent des \textbf{acteurs discursifs}, influençant la vérité numérique. D’où l’importance d’une approche \emph{techno-éthique} de l’investigation numérique.

\bigskip
\section*{8. L’Investigation Numérique dans la Police Judiciaire Moderne}
L’investigation numérique est aujourd’hui un pilier de la police judiciaire. Ses apports majeurs :
\begin{itemize}[noitemsep]
  \item \textbf{Récupération de traces invisibles} : fichiers supprimés, métadonnées, logs.
  \item \textbf{Reconstitution chronologique} : établissement de timelines précises.
  \item \textbf{Attribution des faits} : corrélation IP, journaux, signatures d’appareils.
\end{itemize}
Elle s’appuie sur la \textbf{méthodologie ACPO (Association of Chief Police Officers)} :
\begin{enumerate}[noitemsep]
  \item Ne jamais altérer les données originales.
  \item Documenter toute action effectuée.
  \item Maintenir la chaîne de custody.
  \item Garantir la reproductibilité des analyses.
\end{enumerate}
L’investigation numérique devient ainsi une \textbf{science probatoire} au cœur de la justice digitale.

\bigskip
\hrule
\bigskip

\begin{center}
\textbf{--- Fin du résumé technique des exposés ---}
\end{center}

\end{document}
