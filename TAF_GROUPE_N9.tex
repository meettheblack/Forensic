
\documentclass[12pt,a4paper]{article}
\usepackage[utf8]{inputenc}
\usepackage[T1]{fontenc}
\usepackage[french]{babel}
\usepackage{geometry}
\geometry{margin=1in}
\usepackage{setspace}
\usepackage{titlesec}
\usepackage{hyperref}
\usepackage{graphicx}
\usepackage{array}
\usepackage{booktabs}

\title{\textbf{Investigation Numérique : Les 10 cas de hacking en Afrique (2015-2025)}}
\author{%
Minyanda Yongui José Loïc\thanks{Matricule : 22P071} \and
Fouda Casimir\thanks{Matricule : 22P043} \and
Nna Francis Emmanuel Roméo\thanks{Matricule : 22P056}%
}
\date{}

\begin{document}

\maketitle
\tableofcontents
\newpage

\section{Introduction}
Depuis une décennie, l’Afrique s’est engagée dans une révolution numérique sans précédent. L’essor des technologies de l’information, la digitalisation des services publics et l’émergence des fintechs ont transformé les économies africaines. Mais cette ouverture numérique s’est accompagnée d’une hausse vertigineuse des cyberattaques.

Selon INTERPOL (2024), le continent enregistre désormais plus de 3 000 attaques par semaine et par organisation, soit une augmentation de 300\% en dix ans. Ces attaques touchent des entreprises privées, des administrations publiques, des ONG, des universités et même des infrastructures stratégiques.

Dans ce contexte, l’investigation numérique apparaît comme un pilier essentiel pour comprendre, documenter et prévenir les attaques. Elle s’appuie sur la collecte, l’analyse et la présentation des preuves numériques dans un cadre légal et scientifique.

Ce travail présente dix cas africains emblématiques d’hacking survenus entre 2015 et 2025, analysés selon quatre critères : la taille, le type d’entreprise, le volume de données affectées et les conséquences financières.

\section{Contexte général de la cybersécurité en Afrique}
La cybersécurité africaine se trouve aujourd’hui à un carrefour stratégique. L’accélération de la numérisation touche tous les secteurs : télécommunications, énergie, santé, administration, éducation, transport et finance. Cependant, la plupart des pays du continent manquent encore d’une infrastructure solide pour protéger leurs systèmes critiques.

Les principales menaces observées sont : les ransomwares, les fraudes au mobile money, l’espionnage numérique, les attaques DDoS et les campagnes de désinformation. Des progrès sont cependant notables avec la mise en place de CERTs et de lois cybernétiques dans des pays comme le Maroc, le Nigéria, l’Afrique du Sud et le Cameroun.

\section{Méthodologie et critères d’analyse}
L’investigation numérique suit cinq étapes fondamentales : identification, collecte, préservation, analyse et rapport.  
Les critères retenus pour évaluer les cas sont :
\begin{itemize}
    \item Taille de l’attaque
    \item Type d’organisation ciblée
    \item Volume de données touchées
    \item Impact financier et réputationnel
\end{itemize}

\section{Les dix cas africains les plus emblématiques d’hacking}
\subsection*{Cas 1 - Ransomware sur Transnet (Afrique du Sud, 2021)}
Entreprise publique de logistique victime d’une attaque par ransomware ayant paralysé les ports de Durban, Cape Town et Ngqura. Environ 7 To de données chiffrées et 60 millions USD de pertes.

\subsection*{Cas 2 - Breach de la CNSS (Maroc, 2025)}
Exfiltration de données de 2 millions de salariés et 500 000 entreprises. Données personnelles compromises suite à une attaque revendiquée par un groupe algérien.

\subsection*{Cas 3 - Attaque sur Eneo (Cameroun, 2024)}
Perturbation des systèmes de facturation et d’achat prépayé. Données clients et transactions affectées, pertes estimées à plusieurs centaines de millions FCFA.

\subsection*{Cas 4 - Attaque par GhostLocker 2.0 (Égypte, 2024)}
Attaque coordonnée sur 30 entreprises par le groupe GhostSec. Données industrielles et stratégiques volées, pertes d’environ 20 millions USD.

\subsection*{Cas 5 - Scandale Pegasus (Maroc, 2020–2021)}
Affaire d’espionnage par le logiciel Pegasus. Surveillance de journalistes, responsables politiques et activistes. Impact majeur sur la confiance institutionnelle.

\subsection*{Cas 6 - Piratage des banques ivoiriennes}
Campagne de phishing visant plusieurs banques (UBA, BNI, NSIA Bank). 6 millions d’euros de pertes directes dues à des accès frauduleux via RAT.

\subsection*{Cas 7 - Cyberattaque sur le système de santé tunisien (2021)}
Attaque DDoS couplée à un ransomware ciblant les hôpitaux publics. Pertes de service estimées à 2,5 millions USD.

\subsection*{Cas 8 - Piratage de la compagnie Ethiopian Airlines (2023)}
Compromission du système mondial de réservation, données personnelles de milliers de passagers volées, pertes estimées à 5 millions USD.

\subsection*{Cas 9 - Fraude au Mobile Money MTN Nigeria (2018)}
8 millions USD détournés via des failles API et des complicités internes.

\subsection*{Cas 10 - Piratage de la Banque Centrale du Nigeria (2015–2016)}
Intrusion prolongée sur les serveurs SWIFT, plusieurs dizaines de millions USD volés. Intervention conjointe du FBI et d’Interpol.

\section{Recommandations}
\begin{enumerate}
    \item Former massivement les experts africains en cybersécurité et forensic.
    \item Créer des CERT régionaux.
    \item Harmoniser les lois autour de la Convention de Malabo.
    \item Favoriser l’hébergement local et le cloud souverain.
    \item Encourager les audits de sécurité réguliers.
    \item Mettre en place des fonds de cyber-résilience pour les PME.
\end{enumerate}

\section{Conclusion}
L’Afrique est à la croisée des chemins : son avenir numérique dépendra de sa capacité à sécuriser ses infrastructures et à former ses talents. Les attaques analysées traduisent la transformation profonde du paysage cybercriminel africain. La cybersécurité doit devenir une responsabilité collective pour garantir un développement numérique durable et souverain.

\end{document}
